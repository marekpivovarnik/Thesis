\chapter*{Introduction}
\addcontentsline{toc}{chapter}{Introduction}

Remote sensing, as the term used nowadays, refers to acquisition of information from airborne and satellite platforms. Acquired information is electromagnetic (EM) radiation emitted or reflected from Earth. It is a powerful tool for observing land surface, atmosphere and oceans which results in many applications in different fields such as meteorology, ecology, global change studies, agriculture, sociology, urban studies and many others \cite{J13}.

Remote sensing activities can be divided into groups according to the different regions of EM spectrum. The boundaries between EM regions are not sharply defined. According to Zemek et al. \cite{Z14} the EM regions can be divided into visible (0.4 – \SI{0.72}{\micro\meter}), near infrared (0.72 – \SI{1.3}{\micro\meter}), short-wave infrared (1.3 – \SI{3}{\micro\meter}), mid-wave infrared (3 – \SI{8}{\micro\meter}), long-wave infrared (8 – \SI{14}{\micro\meter}) and microwave (\SI{1}{\milli\meter}~–~\SI{1}{\meter}). First three mentioned EM regions are mostly used for acquisition of reflected EM radiation. The EM radiation in long-wave infrared region, also widely called \textit{thermal} infrared region, contains radiation, which is mainly emitted. Mid-wave infrared consists of mixture of reflected and emitted EM radiation. Microwave radiation is used by radar systems for active and passive remote sensing. Data acquisition is also limited by atmosphere transmittance, which can be very weak between individual regions.

%Remote sensing can be divided into several categories according to the purpose of use.
%For example in visible and near infrared region of electromagnetic spectrum (\SI{0.4}{} – \SI{1.3}{\micro\meter}) are usually used CCDs (Charge-Coupled Device). However, in thermal region of electromagnetic spectrum (8 – \SI{14}{\micro\meter}) are used MCT (mercury cadmium telluride) detectors or microbolometric detectors. 
According to the number of spectral bands and its widths, sensors are divided into broadband, multispectral and hyperspectral. Broadband sensors are continuously sensitive within the one region of EM spectrum, multispectral sensors consists of few, rather wide, spectral bands and hyperspectral sensors acquire data in many very narrow spectral bands. 

The previous splittings set the scope for the contents of this treatise. The treatise focuses on theory and methodology for temperature and emissivity separation from data acquired by airborne thermal hyperspectral sensor. In other words, the data are acquired by airborne sensor recording emitted EM radiation in region of 8 – \SI{14}{\micro\meter} in several narrow bands. Airborne thermal hyperspectral data offers valuable information, which is useful in fields focused on evapotranspiration \cite{PP12}, vegetation \cite{RC10}, soil moisture \cite{SF12}, mineral mapping \cite{NK14}, urban studies \cite{SO12}, gas plumes identification \cite{PM05} and others.

The first airborne thermal multispectral sensor was developed in 1980 by NASA JPL. This sensor consists of five multispectral bands in thermal region. Currently operational airborne sensors are Airborne Hyperspectral Scanner (AHS 80) and Spatially Enhanced Broad Array Spectrograph System (SEBASS). To our best knowledge, there are currently three commercially available airborne thermal hyperspectral sensors, namely TASI (Itres Ltd., Calgary, Canada), AISA Owl (Specim Ltd., Oulu, Finland) and Hyper-Cam LW (Telops Inc., Quebec, Canada).

The treatise starts with theoretical background of thermal radiation. In the first section are mentioned fundamental laws and principles necessary for correct understanding of properties of data acquired by airborne thermal hyperspectral sensor. The interpretation of such a data is introduced in the second section together with description~of main difficulty in temperature and emissivity separation. Third section offers overview of~currently used algorithms for temperature and emissivity separation as well as methodologies for atmospheric corrections. In the fourth section are briefly mentioned preliminary results on improvement of one of the temperature and emissivity separation algorithms mentioned in third section. The treatise is enclosed with definition of main aims of the doctoral thesis.

