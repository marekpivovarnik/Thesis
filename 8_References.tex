\begin{thebibliography}{10}
\addcontentsline{toc}{chapter}{Bibliography}
%\makeatletter
%\def\@biblabel#1{}
%\let\old@bibitem\bibitem
%\def\bibitem#1{\old@bibitem{#1}\leavevmode\kern-\bibindent}
%\makeatother

%\footnotesize

\section*{Author’s publications}
%%%%%%%%%%%%%%%%%%%%%%%%%%%%%%%%%%%%%%%%%%%%%%%%%%%%%%%%%%%%%%%%%%%%%%%%%%%%%%%%%%%%%%%%%%%%%%%%%%%
% MOJE PUBLIKACE
%\bibitem{hf_ibcst} Holešovský, J., Fusek, M. (2015). Metody analýzy extrémních hodnot a~jejich softwarová implementace. Odesláno k~publikaci.
%%
%\bibitem{hfm_hsj} Holešovský, J., Fusek, M., Blachut, V., Michálek, J. (2015). Comparison of precipitation extremes estimation using parametric and nonparametric methods. \emph{Hydrological Sciences Journal}. Přijato k~publikaci. DOI 10.1080/02626667.2015.1111517.
%%
%\bibitem{hfm1} Holešovský, J., Fusek, M., Michálek, J. (2015). Modelling of precipitation extremes using parametric and nonparametric methods with automated threshold selection. \emph{International Journal of Mathematics and Computers in Simulation} \textbf{9}, 94-102.
%%
%\bibitem{hfm_recko} Holešovský, J., Fusek, M., Michálek, J. (2014). Automated threshold selection for parametric and non-parametric estimates of intensity-duration-frequency curves. \emph{Proceedings of the 1st International Conference on Mathematical Methods \& Computational Techniques in Science \& Engineering, MMCTSE 2014}. Athens, Greece. ISBN 987-1-61804-256-9.
%%
%\bibitem{hfm_mendel} Holešovský, J., Fusek, M., Michálek, J. (2014). Extreme value estimation for correlated observations. \emph{20th International Conference on Soft Computing MENDEL 2014.} Brno, Czech Republic, 359-364.
%%
%\bibitem{hp_pdmu} Holešovský, J., Popela, P. (2012). Stochastic Extensions of a~Traffic Assignment Problem. \emph{XX International Conference PDMU-2012: Problems of Decision Making under Uncertainties, Proceeding - Applied Papers.} Brno, Czech Republic, 61-70. ISBN 978-80-7231-897-1.
%%
%\bibitem{hpr_mendel} Holešovský, J., Popela, P., Roupec, J. (2013). Disruption in Congested Networks. \emph{Proceedings of 19th International Conference on Soft Computing MENDEL 2013.} Brno, Czech Republic, 191-196. ISBN 978-80-214-4755-4.
%%
%\bibitem{hz_ryby} Olejníčková, Z., Holešovský, J., Vávrová, M., Králová, Z., Michálek, J. (2014). Methylmercury in tissues of fish from the Svratka River, the Czech Republic. \emph{Fresenius Environmental Bulletin} \textbf{23}(12b), 3319-3324.
%%
%\bibitem{hs_zizaly} Škarková, P., Zlámalová Gargošová, H., Holešovský, J., Vávrová, M., Michálek, J., Olejníčková, Z. (2015). Application of statistical methods for ecotoxicological data evaluation. \emph{Fresenius Environmental Bulletin} \textbf{24}(5), 1692-1698.

\subsection*{Scientifically less trustable references}

%%%%%%%%%%%%%%%%%%%%%%%%%%%%%%%%%%%%%%%%%%%%%%%%%%%%%%%%%%%%%%%%%%%%%%%%%%%%%%%%%%%%%%%%%%%%%%%%%%%

\section*{Other references}

%\addcontentsline{toc}{section}{References}

\bibitem{B15}
B.~Eng, personal communication, 2015.

\bibitem{BH09} BALDRIDGE, A.M., HOOK, S.J., GROVE, C.I. and RIVERA, G. The ASTER spectral library version 2.0. \textit{Remote Sensing of Environment}. 2009, \textbf{113}(4), 711-715. DOI: 10.1016/j.rse.2008.11.007.

\bibitem{BP96} BARDUCCI, A. and PIPPI, I. Temperature and emissivity retrieval from remotely sensed images using the "Grey body emissivity" method. \textit{IEEE Transactions on Geoscience and Remote Sensing}. 1996, \textbf{34}(3), 681-695. DOI: 10.1109/36.499748.

\bibitem{BG06} BERK, A., ANDERSON, G.P., ACHARYA, P.K., BERNSTEIN, L.S., MURATOV, L., LEE, J., FOX, M., ADLER-GOLDEN, S.M., CHETWYND, J.H., HOKE, JR. M.L., LOCKWOOD, R.B., GARDNER, J.A., COOLEY, T.W., BOREL, C.C., LEWIS, P.E. and SHETTLE, E.P. MODTRAN5: 2006 update. \textbf{In:} \textit{Algorithms and Technologies for Multispectral, Hyperspectral, and Ultraspectral Imagery XII}. Orlando (Kissimmee), FL, 2006, F2331--F2331. DOI: 10.1117/12.665077.

\bibitem{B11} BORBAS, E., SEEMANN, S.W., KERN, A., MOY, L., LI, J., GUMLEY, L. and MENZEL, W.P. \textit{MODIS Atmospheric Profile Retrieval - ATBD} [online]. 2011. [cit. 2016-08-10]. Available at: http://modis-atmos.gsfc.nasa.gov/\_docs/MOD07\_atbd\_v7\linebreak\_April2011.pdf

\bibitem{B98} BOREL, C. Surface emissivity and temperature retrieval for a hyperspectral sensor. \textbf{In:} \textit{IGARSS '98. Sensing and Managing the Environment. 1998 IEEE International Geoscience and Remote Sensing. Symposium Proceedings. (Cat. No.98CH36174).} Seattle, WA: IEEE, 1998, 546-549. DOI: 10.1109/IGARSS.1998.702966.

\bibitem{B08} BOREL, C. Error analysis for a temperature and emissivity retrieval algorithm for hyperspectral imaging data. \textit{International Journal of Remote Sensing} [online]. 2008, \textbf{29}(17-18), 5029-5045. DOI: 10.1080/01431160802036540.

\bibitem{BH01} BRADSHAW, A.D. and HÜTTL, R.F. Future minesite restoration involves a broader approach. Ecological Engineering. 2001, \textbf{17}(2-3), 87-90. DOI: 10.1016/S0925-8574(00)00149-X. ISSN 09258574.

\bibitem{CC98} CHEVALLIER, F., CHÉRUY, F., SCOTT, N.A. and CHÉDIN, A. A neural network approach for a fast and accurate computation of a longwave radiative budget. \textit{Journal of Applied Meteorology}. 1998, \textbf{37}(11), 1385-1397. DOI: 10.1175/1520-0450(1998)037<1385:ANNAFA>2.0.CO;2.

\bibitem{CS85} CHÉDIN, A., SCOTT, N.A., WAHICHE, C. and MOULINIER, P. The improved initialization inversion method: a high resolution physical method for temperature retrievals from satellites of the TIROS-N series. \textit{Journal of Climate and Applied Meteorology}. 1985, \textbf{24}(2), 128-143. DOI: 10.1175/1520-0450(1985)024<0128:TIIIMA>2.0.CO;2

\bibitem{CB00} CHRISTENSEN, P.R., BANDFIELD, J.L., HAMILTON, V.E., HOWARD, D.A., LANE, M.D., PIATEK, J.L., RUFF, S.W. and STEFANOV, W.L. A thermal emission spectral library of rock-forming minerals. \textit{Journal of Geophysical Research: Planets}. 2000, \textbf{105}(E4), 9735-9739. DOI: 10.1029/1998JE000624.

\bibitem{CS16} CLARK, R.N., SWAYZE, G.A., WISE, R., LIVO, E., HOEFEN, T., KOKALY, R., SUTLEY, S.J. \textit{USGS digital spectral library splib06a} [online]. U.S. Geological Survey, Digital Data Series 231, 2007. [cit. 2016-07-12]. Available at: http://speclab.cr.usgs.gov/spectral.lib06

\bibitem{CC07} COLL, C., CASELLES, V., VALOR, E., NICLÒS, R., SÁNCHEZ, J.M., GALVE, J.M. and MIRA, M. Temperature and emissivity separation from ASTER data for low spectral contrast surfaces. \textit{Remote Sensing of Environment}. 2007, \textbf{110}(2), 162-175. DOI: 10.1016/j.rse.2007.02.008. 

\bibitem{FM05} FERNÁNDEZ-RENAU, A., MEYNART, R., NEECK, S.P., GÓMEZ, J.A., DE MIGUEL, E. and SHIMODA, H. The INTA AHS system. \textbf{In:} \textit{SPIE Proceedings}. Bruges: SPIE, 2005, 471-478. DOI: 10.1117/12.629440.

\bibitem{FK05} FROUZ, J., KRIŠTŮFEK, V., BASTL, J., KALČÍK, J. and VAŇKOVÁ, H. Determination of Toxicity of Spoil Substrates After Brown Coal Mining Using a Laboratory Reproduction Test with Enchytraeus crypticus (Oligochaeta). \textit{Water, Air, \& Soil Pollution}. 2005, \textbf{162}(1-4), 37-47. DOI: 10.1007/s11270-005-5991-y. 

\bibitem{FL13} FROUZ, J., LIVEČKOVÁ, M., ALBRECHTOVÁ, J., CHROŇÁKOVÁ, A., CAJTHAML, T., PIŽL, V., HÁNĚL, L., STARÝ, J., BALDRIAN, P., LHOTÁKOVÁ, Z., ŠIMÁČKOVÁ, H. and CEPÁKOVÁ, Š. Is the effect of trees on soil properties mediated by soil fauna? A case study from post-mining sites. \textit{Forest Ecology and Management}. 2013, \textbf{309}, 87-95. DOI: 10.1016/j.foreco.2013.02.013.

\bibitem{GV13} GARCIA-SANTOS, V., VALOR, E., CASELLES, V., MIRA, M., GALVE, J.M. a COLL, C. Evaluation of Different Methods to Retrieve the Hemispherical Downwelling Irradiance in the Thermal Infrared Region for Field Measurements. \textit{IEEE Transactions on Geoscience and Remote Sensing}. 2013, \textbf{51}(4), 2155-2165. DOI: 10.1109/TGRS.2012.2209891.

\bibitem{G86} GILLESPIE, A.R., \textit{Lithologic mapping of silicate rocks using TIMS} [online]. Jet Propulsion Lab., California Inst. of Tech., Pasadena, CA, United States, 1986. [cit. 2016-08-10]. Available at: http://ntrs.nasa.gov/search.jsp?R=19870007685

\bibitem{GA11} GILLESPIE, A.R., ABBOTT, E.A., GILSON, L., HULLEY, G., JIMÉNEZ-MUÑOZ, J.C. and SOBRINO, J.A. Residual errors in ASTER temperature and emissivity standard products AST08 and AST05. \textit{Remote Sensing of Environment}. 2011, \textbf{115}(12), 3681-3694. DOI: 10.1016/j.rse.2011.09.007.

\bibitem{GR99} GILLESPIE, A.R., ROKUGAWA, S., HOOK, S., MATSUNAGA, T. and KAHLE, A.B. \textit{Temperature/Emissivity Separation Algorithm Theoretical Basis Document, Version 2.4} [online]. Pasadena: Jet Propulsion Laboratory, 1999. [cit. 2016-01-19]. Available at: http://eospso.nasa.gov/sites/default/files/atbd/atbd-ast-05-08.pdf 

\bibitem{GR98} GILLESPIE, A.R., ROKUGAWA, S., MATSUNAGA, T., COTHERN, J.S., HOOK, S. and KAHLE, A.B. A temperature and emissivity separation algorithm for Advanced Spaceborne Thermal Emission and Reflection Radiometer (ASTER) images. \textit{IEEE Transactions on Geoscience and Remote Sensing}. 1998, \textbf{36}(4), 1113-1126. DOI: 10.1109/36.700995.

\bibitem{GG00} GU, D., GILLESPIE, A.R., KAHLE, A.B. and PALLUCONI, F.D. Autonomous atmospheric compensation (AAC) of high resolution hyperspectral thermal infrared remote-sensing imagery. \textit{IEEE Transactions on Geoscience and Remote Sensing}. 2000, \textbf{38}(6), 2557-2570. DOI: 10.1109/36.885203.

\bibitem{GG06} GUSTAFSON, W.T., GILLESPIE, A.R. and YAMADA, G.J. Revisions to the ASTER temperature/emissivity separation algorithm. \textbf{In:} \textit{Second Recent Advances in Quantitative Remote Sensing}. Valencia, 2006, 770-775.

\bibitem{HF14} HANUŠ, J., FABIÁNEK, T., KAPLAN, V. and HOMOLOVÁ, L. Flying laboratory of imaging systems (FLIS) at CzechGlobe. \textbf{In:} \textit{SGEM2014 Conference Proceedings}. 2014, 177-182. DOI: 10.5593/SGEM2014/B23/S10.022. ISBN 978-619-7105-12-4. ISSN 1314-2704.

\bibitem{HL97} HAMOUZ, K., LACHMAN, J., PIVEC, V. and ORSÁK, M. The effect of the conditions of cultivation on the content of polyphenol compounds in the potato varieties Agria and Karin. \textit{Rostlinná Výroba}. 1997, \textbf{43}, 541-546.

\bibitem{HK96} HOOK, S.J. and KAHLE, A.B. The micro fourier transform interferometer (uFTIR) — A new field spectrometer for acquisition of infrared data of natural surfaces. \textit{Remote Sensing of Environment}. 1996, \textbf{56}(3), 172-181. DOI: 10.1016/0034-4257(95)00231-6.

\bibitem{HJ98} HORTON, K.A., JOHNSON, J.R. and LUCEY, P.G. Infrared Measurements of Pristine and Disturbed Soils 2. Environmental Effects and Field Data Reduction. Remote Sensing of Environment. 1998, \textbf{64}(1), 47-52. DOI: 10.1016/S0034-4257(97)00167-3.

\bibitem{H11} HOWELL, J.R. \textit{Thermal radiation heat transfer}. 5th ed. Boca Raton: CRC Press, c2011. 957. ISBN 978-1-4398-0533-6.

\bibitem{HH11} HULLEY, G. and HOOK, S.J. Generating Consistent Land Surface Temperature and Emissivity Products Between ASTER and MODIS Data for Earth Science Research. \textit{IEEE Transactions on Geoscience and Remote Sensing}. 2011, \textbf{49}(4), 1304-1315. DOI: 10.1109/TGRS.2010.2063034.

\bibitem{HH11-2}  HULLEY, G. and HOOK, S.J. \textit{HyspIRI Level-2 Thermal Infrared (TIR) Land Surface Temperature and Emissivity Algorithm Theoretical Basis Document} [online]. Pasadena, California: Jet Propulsion Laboratory, California Institute of Technology, 2011. [cit. 2016-01-19]. Available at: https://hyspiri.jpl.nasa.gov/downloads/Algorithm\_Theoretical\_Basis/HyspIRI\_L2\linebreak\_Surface\_Temperature\_Emissivity\_JPL\_Pub\_11-5\_10102011.pdf

\bibitem{software:SparCal} ITRES. SparCal [software]. [access  1.3.2016]. Available at: http://www.itres.com\linebreak/supporting-products/

\bibitem{software:RCX} ITRES. RCX [software]. [access 1.3.2016]. Available at: http://www.itres.com\linebreak/supporting-products/

\bibitem{software:GCSS} ITRES. GCSS [software]. [access 1.3.2016]. Available at: http://www.itres.com\linebreak/supporting-products/

\bibitem{JC09} JIMÉNEZ-MUÑOZ, J.C., CRISTOBAL, J., SOBRINO, J.A., SORIA, G., NINYEROLA, M. and PONS, X. Revision of the Single-Channel Algorithm for Land Surface Temperature Retrieval From Landsat Thermal-Infrared Data. \textit{IEEE Transactions on Geoscience and Remote Sensing}. 2009, \textbf{47}(1), 339-349. DOI: 10.1109/TGRS.2008.2007125.

\bibitem{JS12} JIMÉNEZ-MUÑOZ, J.C., SOBRINO, J.A. and GILLESPIE, A.R. Surface Emissivity Retrieval From Airborne Hyperspectral Scanner Data: Insights on Atmospheric Correction and Noise Removal. \textit{IEEE Geoscience and Remote Sensing Letters}. 2012, \textbf{9}(2), 180-184. DOI: 10.1109/LGRS.2011.2163699.

\bibitem{JS14} JIMÉNEZ-MUÑOZ, J.C., SOBRINO, J.A., MATTAR, C., HULLEY, G. and GOTTSCHE, F.-M. Temperature and Emissivity Separation From MSG/SEVIRI Data. \textit{IEEE Transactions on Geoscience and Remote Sensing}. 2014, \textbf{52}(9), 5937-5951. DOI: 10.1109/TGRS.2013.2293791. 

\bibitem{J03} JONES, J.B. \textit{Agronomic handbook: management of crops, soils, and their fertility}. Boca Raton, Fla.: CRC Press, c2003. ISBN 0849308976.

\bibitem{K60} KIRCHHOFF, G. Ueber das Verhältniss zwischen dem Emissionsvermögen und dem Absorptionsvermögen der Körper für Wärme und Licht. \textit{Annalen der Physik und Chemie}. 1860, \textbf{185}(2), 275-301. DOI: 10.1002/andp.18601850205.

\bibitem{KS14} KOTTHAUS, S., SMITH, T.E.L., WOOSTER, M.J. and GRIMMOND, C.S.B. Derivation of an urban materials spectral library through emittance and reflectance spectroscopy. \textit{ISPRS Journal of Photogrammetry and Remote Sensing}. 2014, \textbf{94}, 194-212. DOI: 10.1016/j.isprsjprs.2014.05.005.

\bibitem{LT13} LI, Z., TANG, B., WU, H., REN, H., YAN, G., WAN, Z., TRIGO, I.F. and SOBRINO, J.A. Satellite-derived land surface temperature: Current status and perspectives. \textit{Remote Sensing of Environment} [online]. 2013, \textbf{131}, 14-37. DOI: 10.1016/j.rse.2012.12.008.

\bibitem{M94} MATSUNAGA, T. A Temperature-Emissivity Separation Method Using an Empirical Relationship between the Mean, the Maximum, and the Minimum of the Thermal Infrared Emissivity Spectrum. \textit{Journal of the Remote Sensing Society of Japan}. 1994, \textbf{14}(3), 230–241. DOI: 10.11440/rssj1981.14.230

\bibitem{MB02} MUSHKIN, A., BALICK, L.K., and GILLESPIE, A.R. Temperature/emissivity separation of MTI data using the Terra/ASTER TES algorithm. \textbf{In:} \textit{Algorithms and Technologies for Multispectral, Hyperspectral, and Ultraspectral Imagery VIII}. Orlando, FL: SPIE, 2002, 328-337. DOI: 10.1117/12.478764.

\bibitem{NK14} NOTESCO, G., KOPAČKOVÁ, V., ROJÍK, P., SCHWARTZ, G., LIVNE, I. and DOR, E. Mineral Classification of Land Surface Using Multispectral LWIR and Hyperspectral SWIR Remote-Sensing Data. A Case Study over the Sokolov Lignite Open-Pit Mines, the Czech Republic. \textit{Remote Sensing}. 2014, \textbf{6}(8), 7005-7025. DOI: 10.3390/rs6087005.

\bibitem{PC14} PASCUCCI, S., CASA, R., BELVISO, C., PALOMBO, A., PIGNATTI, S. and CASTALDI, F. Estimation of soil organic carbon from airborne hyperspectral thermal infrared data: a case study. \textit{European Journal of Soil Science}. 2014, \textbf{65}(6), 865-875. DOI: 10.1111/ejss.12203.

\bibitem{PP12} PIPIA, L., PEREZ, F., TARDA, A., MARTINEZ, L. and ARBIOL, R. Simultaneous usage of optic and thermal hyperspectral sensors for crop water stress characterization. \textbf{In:} \textit{IEEE International Geoscience and Remote Sensing Symposium}. Munich: IEEE, 2012, 6661-6664. DOI: 10.1109/IGARSS.2012.6352071. 

\bibitem{P00} PLANCK, M. Zur Theorie des Gesetzes der Energieverteilung im Normalspektrum. \textit{Verhandlungen der Deutschen Physikalischen Gesellschaft}. 1900, \textbf{2}(17), p. 237-245.

\bibitem{RC10} RIBEIRO DA LUZ, B. and CROWLEY, J.K. Identification of plant species by using high spatial and spectral resolution thermal infrared (8.0–$\SI{13.5}{\micro\meter}$) imagery. \textit{Remote Sensing of Environment}. 2010, \textbf{114}(2), 404-413. DOI: 10.1016/j.rse.2009.09.019.

\bibitem{RS02} RICHTER, R. a SCHLÄPFER, D. Geo-atmospheric processing of airborne imaging spectrometry data. Part 2: Atmospheric/topographic correction. \textit{International Journal of Remote Sensing}. 2002, \textbf{23}(13), 2631-2649. DOI: 10.1080/01431160110115834.

\bibitem{SG09} SABOL, Jr., D.E., GILLESPIE, A.R., ABBOTT, E. and YAMADA, G. Field validation of the ASTER Temperature–Emissivity Separation algorithm. \textit{Remote Sensing of Environment}. 2009, \textbf{113}(11), 2328-2344. DOI: 10.1016/j.rse.2009.06.008.

\bibitem{SW91} SALISBURY, J.W., WALTER, L.S., VERGO, N., D'ARIA, D.M. \textit{Infrared (2.1-25 um) spectra of minerals}. Baltimore: John Hopkins University Press, 1991. ISBN 0801844398.

\bibitem{SL05} SHUKLA, M. K. a R. LAL. Temporal Changes in Soil Organic Carbon Concentration and Stocks in Reclaimed Minesoils of Southeastern Ohio. \textit{Soil Science}. 2005, \textbf{170}(12), 1013-1021. DOI: 10.1097/01.ss.0000187354.62481.91.

\bibitem{SW98} SNYDER, W. C., WAN, Z., ZHANG, Y. and FENG, Y.-Z. Classification-based emissivity for land surface temperature measurement from space. \textit{International Journal of Remote Sensing}. 1998, \textbf{19}(14), 2753-2774. DOI: 10.1080/014311698214497.

\bibitem{SF12} SOBRINO, J.A., FRANCH, B., MATTAR, C., JIMÉNEZ-MUÑOZ, J.C. and CORBARI, C. A method to estimate soil moisture from Airborne Hyperspectral Scanner (AHS) and ASTER data: Application to SEN2FLEX and SEN3EXP campaigns. \textit{Remote Sensing of Environment}. 2012, \textbf{117}, 415–428. DOI: 10.1016/j.rse.2011.10.018

\bibitem{SJ07} SOBRINO, J.A., JIMÉNEZ-MUÑOZ, J.C., BALICK, L., GILLESPIE, A.R., SABOL, D., and GUSTAFSON, W. Accuracy of ASTER Level-2 thermal-infrared Standard Products of an agricultural area in Spain. \textit{Remote Sensing of Environment}. 2007, \textbf{106}(2), 146-153. DOI: 10.1016/j.rse.2006.08.010. ISSN 00344257.

\bibitem{SJ02} SOBRINO, J.A., JIMÉNEZ-MUÑOZ, J.C., LABED-NACHBRAND, J. and NERRY, F. Surface emissivity retrieval from Digital Airborne Imaging Spectrometer data. \textit{Journal of Geophysical Research: Atmospheres}. 2002, \textbf{107}(D23). DOI: 10.1029/2002JD002197.

\bibitem{SJ06} SOBRINO, J.A., JIMÉNEZ-MUÑOZ, J.C., ZARCO-TEJADA, P.J., SEPULCRE-CANTÓ, G. and DE MIGUEL, E. Land surface temperature derived from airborne hyperspectral scanner thermal infrared data. \textit{Remote Sensing of Environment}. 2006, \textbf{102}(1-2), 99-115. DOI: 10.1016/j.rse.2006.02.001.

\bibitem{SM09} SOBRINO, J.A., MATTAR, C., PARDO, P., JIMÉNEZ-MUÑOZ, J.C., HOOK, S.J., BALDRIDGE, A. and IBAÑEZ, R. Soil emissivity and reflectance spectra measurements. \textit{Applied Optics}. 2009, \textbf{48}(19), 3664-3670. DOI: 10.1364/AO.48.003664.

\bibitem{SO12} SOBRINO, J.A., OLTRA-CARRIÓ, R., JIMÉNEZ-MUÑOZ, J.C., JULIEN, Y., SÒRIA, G., FRANCH, B. and MATTAR, C. Emissivity mapping over urban areas using a classification-based approach: Application to the Dual-use European Security IR Experiment (DESIREX). \textit{International Journal of Applied Earth Observation and Geoinformation}. 2012, \textbf{18}, 141–147. DOI: 10.1016/j.jag.2012.01.022.

\bibitem{SR00} SOBRINO, J.A. and RAISSOUNI, N. Toward remote sensing methods for land cover dynamic monitoring: Application to Morocco. \textit{International Journal of Remote Sensing}. 2000, \textbf{21}(2), 353-366. DOI: 10.1080/014311600210876.

\bibitem{SS07} SÒRIA, G. and J.A. SOBRINO. ENVISAT/AATSR derived land surface temperature over a heterogeneous region. \textit{Remote Sensing of Environment}. 2007, \textbf{111}(4), 409-422. DOI: 10.1016/j.rse.2007.03.017.

\bibitem{T05} TONOOKA, H. Accurate atmospheric correction of ASTER thermal infrared imagery using the WVS method. \textit{IEEE Transactions on Geoscience and Remote Sensing}. 2005, \textbf{43}(12), 2778-2792. DOI: 10.1109/TGRS.2005.857886.

\bibitem{TP01} TONOOKA, H. and PALLUCONI, D. Verification of the ASTER/TIR atmospheric correction algorithm based on water surface emissivity retrieved. \textbf{In:} \textit{Proceedings of the Society of Photo-optical Instrumentation Engineers}. San Diego, CA: SPIE, 2001, 51-58. DOI: 10.1117/12.455143.

\bibitem{TP05} TONOOKA, H. and PALLUCONI, D. Validation of ASTER/TIR standard atmospheric correction using water surfaces. \textit{IEEE Transactions on Geoscience and Remote Sensing}. 2005, \textbf{43}(12), 2769-2777. DOI: 10.1109/TGRS.2005.857883.

\bibitem{TP05-2}  TONOOKA, H., PALLUCONI, D., HOOK, S.J. and MATSUNAGA, T. Vicarious calibration of ASTER thermal infrared bands. \textit{IEEE Transactions on Geoscience and Remote Sensing}. 2005, \textbf{43}(12), 2733-2746. DOI: 10.1109/TGRS.2005.857885.

\bibitem{UL06} USSIRI, D.A.N., LAL, R. and JACINTHE, P.-A. Post-reclamation Land Use Effects on Properties and Carbon Sequestration in Minesoils of Southeastern Ohio. \textit{Soil Science}. 2006, \textbf{171}(3), 261-271. DOI: 10.1097/01.ss.0000199702.68654.1e.

\bibitem{VD01} Edited by VAN DER MEER, F.D. and DE JONG, S.M. \textit{Imaging Spectrometry: Basic Principles and Prospective Applications}. Dordrecht: Kluwer Academic Publishers, 2001. ISBN 978-0-306-47578-8.

\bibitem{VF13} VINDUŠKOVÁ, O. a FROUZ, J. Soil carbon accumulation after open-cast coal and oil shale mining in Northern Hemisphere: a quantitative review. \textit{Environmental Earth Sciences}. 2013, \textbf{69}(5), 1685-1698. DOI: 10.1007/s12665-012-2004-5. 

\bibitem{W08} WAN, Z. New refinements and validation of the MODIS Land-Surface Temperature/Emissivity products. \textit{Remote Sensing of Environment}. 2008, \textbf{112}(1), 59-74. DOI: 10.1016/j.rse.2006.06.026. 

\bibitem{WW11} WANG, N., WU, H., NERRY, H., LI, C. and LI, Z. Temperature and Emissivity Retrievals From Hyperspectral Thermal Infrared Data Using Linear Spectral Emissivity Constraint. \textit{IEEE Transactions on Geoscience and Remote Sensing}. 2011, \textbf{49}(4), 1291-1303. DOI: 10.1109/TGRS.2010.2062527.

\bibitem{WX11} WANG, H., XIAO, Q., LI, H. and ZHONG, B. Temperature and emissivity separation algorithm for TASI airborne thermal hyperspectral data. \textbf{In:} \textit{2011 International Conference on Electronics, Communications and Control (ICECC)}. Ningbo: IEEE, 2011, 1075-1078. DOI: 10.1109/icecc.2011.6066288.

\bibitem{Y02} YOUNG, S.J. An in-scene method for atmospheric compensation of thermal hyperspectral data. \textit{Journal of Geophysical Research}. 2002, \textbf{107}(D24). DOI: 10.1029/2001jd001266.

\bibitem{Z14} Edited by ZEMEK, F.. \textit{Airborne remote sensing: theory and practice in assessment of terrestrial ecosystems.} Brno: Global Change Research Institute CAS, 2014. 159. ISBN 978-80-87902-05-9.

% ASTER applications

\bibitem{VV12} VAN DER MEER, F.D., VAN DER WERFF, H.M.A., VAN RUITENBEEK, F.J.A., et al. Multi- and hyperspectral geologic remote sensing: A review. \textit{International Journal of Applied Earth Observation and Geoinformatio}n. 2012, \textbf{14}(1), 112-128. DOI: 10.1016/j.jag.2011.08.002.

\bibitem{PA04} PIERI, D. and ABRAMS, M. ASTER watches the world's volcanoes: a new paradigm for volcanological observations from orbit. \textit{Journal of Volcanology
  and Geothermal Research}. 2004, \textbf{135}(1-2), 13-28. DOI: 10.1016/j.jvoleores.2003.12.018

\bibitem{FB12} FOSTER, L.A., BROCK, B.W., CUTLER, M.E.J. and DIOTRI, F. A physically based method for estimating supraglacial debris thickness from thermal band remote-sensing data. \textit{Journal of Glaciology}. 2012, \textbf{58}(210), 677-691. DOI: 10.3189/2012JoG11J194. 

\bibitem{SR10} SCHEIDT, S., RAMSEY, M. and LANCASTER, N. Determining soil moisture and sediment availability at White Sands Dune Field, New Mexico, from apparent thermal inertia data. \textit{Journal of Geophysical Research: Earth Surface}. 2010, \textbf{115}(F2). DOI: 10.1029/2009JF001378.

\bibitem{WR11} WENG, Q., RAJASEKAR, U. and HU, X. Modeling Urban Heat Islands and Their Relationship With Impervious Surface and Vegetation Abundance by Using ASTER Images. \textit{IEEE Transactions on Geoscience and Remote Sensing}. 2011, \textbf{49}(10), 4080-4089. DOI: 10.1109/TGRS.2011.

\bibitem{FS08} FRENCH, A., SCHMUGGE, T., RITCHIE, J., HSU, A., JACOB, F. and OGAWA, K. Detecting land cover change at the Jornada Experimental Range, New Mexico with ASTER emissivities. \textit{Remote Sensing of Environment}. 2008, \textbf{112}(4), 1730-1748. DOI: 10.1016/j.rse.2007.08.020.

\end{thebibliography}