\chapter{Conclusion}

The TES algorithm is well established and popular for several reasons: it retrieves temperature and emissivity of natural surfaces simultaneously without any previous knowledge of surface type and it is widely applicable to range of multispectral and hyperspectral sensors. This suggests that the algorithm is a good benchmark for temperature and emissivity separation. \textcolor{red}{Any improvement to the TES algorithm} can benefit \textcolor{red}{processing of thermal data} from many sources.

This work introduced a module that estimates temperature and emissivity from an approximation of the relationship between brightness temperature and emissivity. The new module replaces the NEM module in the original TES to create an algorithm we call OSTES. The performance of OSTES was firstly tested on a set of simulated data recomputed with respect to ASTER, AHS and TASI response functions. Results show that temperature estimations using OSTES are more accurate and precise than TES for samples with low spectral contrast. It should be noted that this improvement is of modest size when compared to the already accurate results that can be obtained with TES. OSTES and TES perform similarly for samples with a high spectral contrast. The results also reveal that OSTES is less sensitive to  variations in atmospheric conditions.

The OSTES was also \textcolor{red}{compared against the} ASTER standard product AST\_09T over the Caspian Sea and Lake Baikal. By comparing the OSTES results to ASTER standard products AST\_08 (temperature) and AST\_05 (emissivity) we found that temperature retrievals of both algorithms are very close. However, it was also found that temperatures included in AST\_08 product are not consistent with emissivities delivered by AST\_05 product in the sense of (\ref{eq:emissivityComputation}). Thus emissivities were recomputed based on downwelling and land-leaving radiance from AST\_09T and temperature from AST\_08. Comparing all three emissivity retrievals over  {the} Caspian Sea in different seasons shows that emissivity from AST\_05 to be closest and recomputed emissivity to be the furthest from expected sea water emissivity values extracted from ASTER Spectral Library, except in the June  {and September scenes}, which  {are} expected to have the largest water vapour burden in the atmosphere. It is also observed that the AST\_05 emissivities over Lake Baikal exhibit step discontinuities. In the same region OSTES and recomputed emissivities tend to be smoother with OSTES emissivities being closer to expected value of water emissivity. All emissivity retrievals are probably affected by inaccurate atmospheric corrections since none of the obtained spectra had emissivity values close to expected values.

We conclude that improvements in atmospheric compensation will be crucial for further improvements in emissivity results. Thus, further work should be focused on this topic. We believe it is important to validate the performance of future improvements by using data acquired by a variety of multispectral and hyperspectral sensors, such as AHS and TASI. Additional improvements in OSTES will consider modifications of cost function represented in (\ref{eq:costFunction}) and illustrated in Fig. \ref{fig:FunctionCode}. Better approximations of the relationship between brightness temperature and emissivity could result in better temperature and emissvity retrievals. 

Since TES plays important role in the TIR data processing, we suggest to use OSTES instead of TES mainly because of higher precision and accuracy under conditions of low spectral contrast, and because of the consistency between retrieved temperature and emissivity. We hope that improvements introduced by OSTES will help to enhance the quality of temperature and emissivity results.