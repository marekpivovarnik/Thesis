\chapter*{Introduction}
\addcontentsline{toc}{chapter}{Introduction}

Remote sensing refers to acquisition of information without making physical contact. The term, as used nowadays, is mostly used in the context of data acquired from airborne and satellite platforms. Acquired information is electromagnetic (EM) radiation emitted or reflected from the Earth. It is a powerful tool for observing the land surface, atmosphere and oceans, which results in many applications in different fields including meteorology, ecology, global change studies, agriculture, sociology, urban studies and many others \cite{VD01}.

Remote sensing activities can be divided into the groups according to the different regions of EM spectrum which is used. The boundaries between EM regions are not sharply defined. According to the \cite{Z14}, the EM spectrum can be divided into visible (0.4 – \SI{0.72}{\micro\meter}), near infrared (0.72 – \SI{1.3}{\micro\meter}), short-wave infrared (1.3 – \SI{3}{\micro\meter}), mid-wave infrared (3 – \SI{8}{\micro\meter}), long-wave infrared (8 – \SI{14}{\micro\meter}) and microwave (\SI{1}{\milli\meter}~–~\SI{1}{\meter}) region. In the first three mentioned regions EM radiation can be observed which is mainly reflected from the Earth's surface. The EM radiation in long-wave infrared region, also widely referred as the \textit{thermal} infrared (TIR) region, consists mainly of the radiation emitted by the Earth's surface. Mid-wave infrared consists of mixture of reflected and emitted EM radiation. Microwave radiation is sensed by radar systems for active remote sensing. Data acquisition is also limited by atmosphere transmittance, which can be very weak between individual regions.

The sensors used for acquisition of EM radiation are categorised into broadband, multispectral and hyperspectral. Broadband sensors are continuously sensitive within the one region of EM spectrum while multispectral sensors consist of few, rather wide, spectral bands within one region of EM spectrum. Hyperspectral sensors are similar to multispectral, but acquire data in many very narrow and consecutive spectral bands. 

The first airborne thermal multispectral sensor was developed in 1980 by NASA Jet Propulsion Laboratory. This sensor consisted of five multispectral bands in the thermal region. Currently operational airborne sensors are Airborne Hyperspectral Scanner (AHS) and Spatially Enhanced Broad Array Spectrograph System (SEBASS). To our best knowledge, there are currently three commercially available airborne thermal hyperspectral sensors, namely Thermal Airborne Spectrographic Imager (TASI) (Itres Ltd., Calgary, Canada), AISA Owl (Specim Ltd., Oulu, Finland) and Hyper-Cam LW (Telops Inc., Quebec, Canada).

Regarding the the various types of remote sensing data, the focus of this work will be put on processing of image data obtained from multispectral and hyperspectral sensors in the TIR region (i.e. the data are obtained by a sensor acquiring emitted EM radiation in the region of 8 – \SI{14}{\micro\meter} in several spectral bands). This work primarily focuses on processing of airborne thermal hyperspectral data acquired by the TASI sensor, however, other sensor types will be mentioned as well.

Airborne thermal hyperspectral data offer valuable information about the observed objects. Image data of this kind has found application in fields focused on evapotranspiration \cite{PP12}, vegetation \cite{RC10}, soil moisture \cite{SF12}, mineral mapping \cite{NK14}, urban studies \cite{SO12} and gas plumes identification \cite{PM05}. Let us emphasize that the most important quantities derived form airborne thermal hyperspectral data are temperature and emissivity. However, direct derivation of temperature and emissivity by observing radiance in $N$ bands results in $N$ equations but $N+1$ unknowns ($N$ emissivities plus temperature). This problem, separating the contributions of temperature and emissivity to observed radiances, has been the subject of a great deal of research and many methods have been developed to address it \cite{LT13}. 
 
The most widely used spaceborne sensor with multispectral TIR capabilities is the Advanced Spaceborne Thermal Emission and Reflection Radiometer (ASTER). It is part of the NASA's Terra platform, which was launched in December 1999. The temperature and emissivity separation algorithm \cite{GR98}, designated TES, that was developed for the ASTER sensor has since been applied to processing of TIR image data acquired by various airborne and spaceborne, and various multispectral and hyperspectral sensors.

Although the TES algorithm is already capable of producing reasonably accurate results it could be made more robust, precise and widely applicable by reducing the number of the assumptions that it makes. In particular, for surfaces with low spectral contrast TES often produces anomalous emissivity spectra \cite{CC07, SJ07}. These spectra suffer from a large degree of noise, which can be explained by the use of various thresholds included in TES. %In fact, as will be described in Chapter \ref{chap:OSTESValid}, the TES-derived standard kinetic temperature and emissivity data products distributed by the Land Processes Distributed Active Archive Center (LP DAAC) are inconsistent, i.e. they are not related with respect to the radiative transfer equation.

The aims of this work are: 1) enhancing the accuracy and precision of the products generated by the TES algorithm and 2) incorporating a new algorithm to the processing chain applied on image data acquired by TASI sensor. The chapters discussing the aims of the work are preceded by Chapter \ref{chap:TheoreticalBackground} and Chapter \ref{chap:Data}, which introduce fundamental laws of thermal radiation and basic principles of the processing of airborne thermal hyperspectral data. Chapter \ref{chap:Data} describes in detail all pre-processing steps applied to image data acquired by the TASI sensor necessary for initiation of the temperature and emissivity separation procedure. These steps create a pre-processing chain, which will be followed by the temperature and emissivity separation procedure.

Chapter \ref{chap:TES} describes the problem of temperature and emissivity separation and introduces currently used algorithms with emphasis on the TES algorithm. This chapter also introduces the improvement of the TES algorithm, which is referred to as Optimized Smoothing for Temperature and Emissivity Separation (OSTES). The main improvement is accomplished by replacing one of the TES modules with a newly designed one that takes advantage of a relationship between brightness temperature and emissivity.

The results of the OSTES performance testing are described in the Chapter \ref{chap:OSTESValid}. The OSTES algorithm is firstly tested on a set of simulated data representing different natural materials as they would be acquired by various multispectral and hyperspectral sensors. Then it is applied on the ASTER standard land-leaving and downwelling radiance product AST\_09T and the results are compared with the ASTER emissivity and surface kinetic temperature standard products AST\_08 and AST\_05, respectively. Last part of this chapter includes incorporation of the OSTES algorithm to the processing chain of image data acquired by the TASI sensor and then it compares the performance of the OSTES and TES algorithms on image data obtained from TASI.